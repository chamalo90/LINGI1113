La première partie de ce projet consistait à calculer la fréquence exacte de l'horloge du PIC. Ce calcul pouvait être effectué de manière très simple. En effet, le \textit{PIC-MAXI-WEB} possède un temporisateur dont l'oscillateur interne peut être facilement réglé pour lancer une interruption après très exactement une seconde. Il nous a alors suffit de programmer le lancement simultané de deux temporisateurs (Timer0 et Timer1, démarrage à une instruction près), le Timer1 étant celui qui overflow après une seconde. Une fois l'interruption déclenchée, il suffit de regarder la valeur contenue dans l'autre temporisateur. Le Timer0 était réglé de manière à ne pas générer d'interruption. Pour \^{e}tre certain qu'il n'overflow pas pendant la seconde qui s'écoule, il fallait programmer le prescaler du Timer0 sur sa valeur maximum. L'incrémentation du temporisateur ne se fait alors que toutes les 256 instructions. On introduit donc un peu d'incertitude car, lors de l'overflow du Timer1, dans le pire cas, on manquera 255 incrémentations du Timer0. On sait, grâce à la documentation, que la fréquence de l'horloge est de 25MHz. Nous pouvons donc signaler que l'erreur est faible ($\approx 0,001$ MHz). Il faut également multiplier la valeur obtenue par 4 car une instruction se déroule sur quatre coups d'horloge. Les commentaires présents dans le code expliquent l'initialisation des différents registres pas à pas. Un calcul simple permet alors d'obtenir la fréquence du PIC : \\
\begin{center}
$Freq_{PIC} = Value_{TMP_{0}} \times 256 \times 4$
\end{center}
Après avoir effectué ce test, nous avons mesuré que la fréquence du PIC est de 25,350 MHz. On peut donc conclure que notre programme calcule relativement précisément la fréquence de l'horloge. \\

Si vous souhaitez effectuer le test, le programme se situe dans le répertoire \textsc{PIC$\_$ChipFreq}. Nous avons utilisé les fichiers de tests donnés pour commencer le projet et avons modifié le contenu de \textsc{test.c} pour effectuer les opérations demandées. Pour compiler le programme de test, il suffit donc d'utiliser la commande \textsc{make}. Après avoir flashé le programme sur la carte, l'écran LCD donne les informations voulues.
