Nous avons pensé à plusieurs améliorations à réaliser si nous disposions de plus temps : 
\begin{itemize}
\item D'abord, les deux appels systèmes exécutent des fonctions forts similaires. Il serait dès lors possible de ne créer qu'un appel système, pour ne pas surcharger l'OS. En ajoutant un argument dans le message, la fonction se terminerait avant la défragmentation si l'utilisateur ne demande que le nombre de fragments.
\item Ensuite, nos hypothèses sont valides pour un système mono-utilisateur simple. Si des mises à jour de \textsc{Minix} ont lieu, un système multi-utilisateurs pourrait rendre nos fonctions obsolètes. Nous avons cependant veillé à utiliser des méthodes d'allocation qui ne forcent pas l'allocation d'une zone. Il s'agit d'une réutilisation de fonctions utilisées par le système d'exploitation. Si un autre processus veut changer l'attribution des zones, notre fonction ne va pas crasher. Cependant, il est possible que la défragmentation soit moins performante et laisse deux ou plusieurs fragments.
\item Enfin, la défragmentation fonctionne ici pour un fichier. Son pouvoir est assez limité car elle ne cherche même pas à réutiliser les zones allouées au fichier à défragmenter. De plus, il serait également possible d'augmenter le nombre de fragment même après défragmentation. Ici, dès qu'il n'y a pas moyen de mettre le fichier en un fragment, la fonction s'arrête. Il serait sans doute plus performant d'essayer d'au moins diminuer le nombre de fragments.
\end{itemize}
On peut cependant affirmer que malgré quelques comportements à risque, le programme est fiable.